%-----------------------------------------------------------------------------------------------------------------------------------------------%
%	The MIT License (MIT)
%
%	Copyright (c) 2021 Jitin Nair
%
%	Permission is hereby granted, free of charge, to any person obtaining a copy
%	of this software and associated documentation files (the "Software"), to deal
%	in the Software without restriction, including without limitation the rights
%	to use, copy, modify, merge, publish, distribute, sublicense, and/or sell
%	copies of the Software, and to permit persons to whom the Software is
%	furnished to do so, subject to the following conditions:
%	
%	THE SOFTWARE IS PROVIDED "AS IS", WITHOUT WARRANTY OF ANY KIND, EXPRESS OR
%	IMPLIED, INCLUDING BUT NOT LIMITED TO THE WARRANTIES OF MERCHANTABILITY,
%	FITNESS FOR A PARTICULAR PURPOSE AND NONINFRINGEMENT. IN NO EVENT SHALL THE
%	AUTHORS OR COPYRIGHT HOLDERS BE LIABLE FOR ANY CLAIM, DAMAGES OR OTHER
%	LIABILITY, WHETHER IN AN ACTION OF CONTRACT, TORT OR OTHERWISE, ARISING FROM,
%	OUT OF OR IN CONNECTION WITH THE SOFTWARE OR THE USE OR OTHER DEALINGS IN
%	THE SOFTWARE.
%	
%
%-----------------------------------------------------------------------------------------------------------------------------------------------%

%----------------------------------------------------------------------------------------
%	DOCUMENT DEFINITION
%----------------------------------------------------------------------------------------

% article class because we want to fully customize the page and not use a cv template
\documentclass[a4paper,12pt]{article}

%----------------------------------------------------------------------------------------
%	FONT
%----------------------------------------------------------------------------------------

% % fontspec allows you to use TTF/OTF fonts directly
% \usepackage{fontspec}
% \defaultfontfeatures{Ligatures=TeX}

% % modified for ShareLaTeX use
% \setmainfont[
% SmallCapsFont = Fontin-SmallCaps.otf,
% BoldFont = Fontin-Bold.otf,
% ItalicFont = Fontin-Italic.otf
% ]
% {Fontin.otf}

%----------------------------------------------------------------------------------------
%	PACKAGES
%----------------------------------------------------------------------------------------
\usepackage{url}
\usepackage{parskip} 	

%other packages for formatting
\RequirePackage{color}
\RequirePackage{graphicx}
\usepackage[usenames,dvipsnames]{xcolor}
\usepackage[scale=0.9]{geometry}

%tabularx environment
\usepackage{tabularx}

%for lists within experience section
\usepackage{enumitem}

% centered version of 'X' col. type
\newcolumntype{C}{>{\centering\arraybackslash}X} 

%to prevent spillover of tabular into next pages
\usepackage{supertabular}
\usepackage{tabularx}
\newlength{\fullcollw}
\setlength{\fullcollw}{0.47\textwidth}

%custom \section
\usepackage{titlesec}				
\usepackage{multicol}
\usepackage{multirow}

%CV Sections inspired by: 
%http://stefano.italians.nl/archives/26
\titleformat{\section}{\Large\scshape\raggedright}{}{0em}{}[\titlerule]
\titlespacing{\section}{0pt}{10pt}{10pt}

%for publications
\usepackage[style=authoryear,sorting=ynt, maxbibnames=2]{biblatex}

%Setup hyperref package, and colours for links
\usepackage[unicode, draft=false]{hyperref}
\definecolor{linkcolour}{rgb}{0,0.2,0.6}
\hypersetup{colorlinks,breaklinks,urlcolor=linkcolour,linkcolor=linkcolour}
\addbibresource{citations.bib}
\setlength\bibitemsep{1em}

%for social icons
\usepackage{fontawesome5}

%debug page outer frames
%\usepackage{showframe}

%----------------------------------------------------------------------------------------
%	BEGIN DOCUMENT
%----------------------------------------------------------------------------------------
\begin{document}

% non-numbered pages
\pagestyle{empty} 

%----------------------------------------------------------------------------------------
%	TITLE
%----------------------------------------------------------------------------------------

% \begin{tabularx}{\linewidth}{ @{}X X@{} }
% \huge{Your Name}\vspace{2pt} & \hfill \emoji{incoming-envelope} email@email.com \\
% \raisebox{-0.05\height}\faGithub\ username \ | \
% \raisebox{-0.00\height}\faLinkedin\ username \ | \ \raisebox{-0.05\height}\faGlobe \ mysite.com  & \hfill \emoji{calling} number
% \end{tabularx}

\begin{tabularx}{\linewidth}{@{} C @{}}
\Huge{Rob Verheyen} \\[7.5pt]
\href{https://github.com/rbvh}{\raisebox{-0.05\height}\faGithub\ rbvh} \ $|$ \ 
\href{https://www.linkedin.com/in/rob-verheyen-55955a97/}{\raisebox{-0.05\height}\faLinkedin\ rob-verheyen} \ $|$ \ 
\href{https://rbvh.github.io/}{\raisebox{-0.05\height}\faGlobe \ rbvh.github.io} \ $|$ \ 
\href{mailto:r.verheyen@ucl.ac.uk}{\raisebox{-0.05\height}\faEnvelope \ r.verheyen@ucl.ac.uk} \ $|$ \ 
\href{tel:+31 683095896}{\raisebox{-0.05\height}\faMobile \ +31 683095896} \\ [2pt]
Oxford, United Kingdom \\
\end{tabularx}

%----------------------------------------------------------------------------------------
% EXPERIENCE SECTIONS
%----------------------------------------------------------------------------------------

%Interests/ Keywords/ Summary
\section{Knowledge \& Research Interests}
\begin{itemize}[nosep,after=\strut, leftmargin=1em, itemsep=3pt]
    \item[-] Monte Carlo techniques; High-performance computing; Algorithmic efficiency
    \item[-] Deep generative models; Anomaly detection; Variational Autoencoders; Normalizing flows; Variational inference; Graph neural networks; Transformers
\end{itemize}

%Experience
\section{Work Experience}

\begin{tabularx}{\linewidth}{ @{}l r@{} }
\textbf{Postdoctoral researcher} Theoretical high energy physics & \hfill Oct 2019 - present \\[3.75pt]
\href{https://www.hep.ucl.ac.uk/}{\underline{University College London}} / \href{https://www.physics.ox.ac.uk/research/subdepartment/particle-physics}{\underline{University of Oxford (visitor)}}, United Kingdom \\ [3.75pt]
\multicolumn{2}{@{}X@{}}{
\begin{minipage}[t]{\linewidth}
    \begin{itemize}[nosep,after=\strut, leftmargin=1em, itemsep=3pt]
        \item[-] Member of the \href{https://https://gsalam.web.cern.ch/gsalam/panscales/}{PanScales} collaboration, a project involving leading theorists that aim to improve theoretical accuracy and understanding of Monte Carlo event generators.
        \item[-] Research on deep generative models, anomaly detection and event classification with models such as normalizing flows, graph neural networks and transformers.
        \item[-] Experience with high-performance computing and algorithmic efficiency, which are core components of the PanScales project.
    \end{itemize}
\end{minipage}} 
\end{tabularx}

%----------------------------------------------------------------------------------------
%	EDUCATION
%----------------------------------------------------------------------------------------
\section{Education}
\begin{tabularx}{\linewidth}{@{}l  r@{}}	
\textbf{PhD} Theoretical high energy physics & \hfill 2015 - 2019 \\ 
\href{https://www.ru.nl/highenergyphysics/}{\underline{Radboud University Nijmegen}}, the Netherlands \\[3.75pt]
\multicolumn{2}{@{}X@{}}{
\begin{minipage}[b]{\linewidth}
    \begin{itemize}[nosep,after=\strut, leftmargin=1em, itemsep=3pt]
        \item[-] Author and developer in the \href{https://pythia.org}{PYTHIA} collaboration, the leading particle physics Monte Carlo event generator and the most widely-used and cited software in the field.
        \item[-] Research on deep generative models for particle collision events, which was a brand new and emerging field at the time.
        \item[-] Strong emphasis on Monte Carlo algorithms and theoretical calculations used for the simulation of highly-energetic particle collisions.
    \end{itemize}
\end{minipage}} \\ [5pt]

\textbf{MsC} Theoretical high energy physics (\emph{Summa cum laude/with highest honors})& \hfill 2013 - 2015 \\
\href{https://www.ru.nl/highenergyphysics/}{\underline{Radboud University Nijmegen}}, the Netherlands \\[3.75pt]
\multicolumn{2}{@{}X@{}}{
\begin{minipage}[b]{\linewidth}
    \begin{itemize}[nosep,after=\strut, leftmargin=1em, itemsep=3pt]
        \item[-] Research with focus on numerical techniques in calculations for supersymmetric field theory. 
    \end{itemize}
\end{minipage}} \\ [5pt]

\textbf{BsC} Physics and Astronomy (\emph{Cum laude/with honors}) & \hfill 2009 - 2013 \\
\href{https://www.ru.nl/highenergyphysics/}{\underline{Radboud University Nijmegen}}, the Netherlands \\[3.75pt]
\end{tabularx}

%----------------------------------------------------------------------------------------
%	PUBLICATIONS
%----------------------------------------------------------------------------------------
\section{Selected publications}
Full list of 24 publications on \href{https://inspirehep.net/authors/1777870?ui-citation-summary=true}{Inspire} or \href{https://scholar.google.com/citations?user=MRTAm7UAAAAJ&hl=en&oi=ao}{Google Scholar}

\underline{\large{Machine learning}} \\ [5pt]
\textbf{Event Generation and Density Estimation with Surjective Normalizing Flows} \\
R. Verheyen, 
\href{https://scipost.org/10.21468/SciPostPhys.13.3.047}{\underline{SciPost Physics 2022}} [\href{https://arxiv.org/abs/2205.01697}{arXiv}] [\href{https://github.com/rbvh/surflows}{code}]

\textbf{Rare and Different: Anomaly Scores from a Combination of Likelihood and Out-of-distribution Models to Detect New Physics at the LHC} \\
S. Caron, L. Hendriks, R. Verheyen,
\href{https://scipost.org/10.21468/SciPostPhys.12.1.043}{\underline{SciPost Physics 2022}} [\href{https://arxiv.org/pdf/2105.14027.pdf}{arXiv}] [\href{https://github.com/bostdiek/DarkMachines-UnsupervisedChallenge}{code}]

\textbf{Event Generation and Statistical Sampling for Physics with Deep Generative Models and a Density Information Buffer} \\
S. Otten, S. Caron, W. de Swart, M. van Beekveld, L. Hendriks, C. van Leeuwen, D. Podareanu, R. Ruiz de Austri, R. Verheyen, 
\href{https://www.nature.com/articles/s41467-021-22616-z}{\underline{Nature Communications 2021}} [\href{https://arxiv.org/abs/1901.00875}{arXiv}]

\textbf{Phase Space Sampling and Inference from Weighted Events with Autoregressive Flows} \\
B. Stienen, R. Verheyen, 
\href{https://scipost.org/10.21468/SciPostPhys.10.2.038}{\underline{SciPost Physics 2021}} [\href{https://arxiv.org/pdf/2011.13445.pdf}{arXiv}] [\href{https://github.com/rbvh/PhaseSpaceAutoregressiveFlow}{code}]



\underline{\large{Monte Carlo generator development}} \\[5pt]
\textbf{A Comprehensive Guide to the Physics and Usage of PYTHIA 8.3} \\
C. Bierlich, S. Chakraborty, N. Desai, L. Gellersen, I. Helenius, P. Ilten, L. Lönnblad, S. Mrenna, S. Prestel, C. Preuss, T. Sjöstrand, P. Skands, M. Utheim, R. Verheyen, \\
\href{https://scipost.org/SciPostPhysCodeb.8-r8.3}{\underline{SciPost Physics Codebases 2022}} [\href{https://arxiv.org/abs/2203.11601}{arXiv}] [\href{https://pythia.org}{code}]

\textbf{PanScales Parton Showers for Hadron Collisions: Formulation and Fixed-order studies} \\
M. van Beekveld, S. Ferrario Ravasio, G. Salam, A. Soto-Ontoso, G. Soyez, R. Verheyen, \\
\href{https://link.springer.com/article/10.1007/JHEP11(2022)019https://arxiv.org/abs/2205.02237}{\underline{Journal of High Energy Physics 2022}} [\href{https://arxiv.org/abs/2205.02237}{arXiv}]

\textbf{Spin Correlations in Final-state Parton Showers and Jet Observables} \\
A. Karlberg, G. Salam, L. Scyboz, R. Verheyen, \\
\href{https://link.springer.com/article/10.1140/epjc/s10052-021-09378-0}{\underline{The European Physical Journal 2021}} [\href{https://arxiv.org/pdf/2103.16526.pdf}{arXiv}]

\textbf{Competing Sudakov Veto Algorithms} \\
R. Kleiss, R. Verheyen,
\href{https://link.springer.com/article/10.1140/epjc/s10052-016-4231-5}{\underline{The European Physical Journal 2016}} [\href{https://arxiv.org/pdf/1605.09246.pdf}{arXiv}] [\href{https://github.com/rbvh/Veto-Algorithm-Toy-Shower}{code}]



% \textbf{The Dark Machines Anomaly Score Challenge: Benchmark Data and Model Independent Event Classification for the Large Hadron Collider} \\
% T. Aarrestad, M. van Beekveld, M. Bona, A. Boveia, S. Caron, J.Davies, A. De Simone, C. Doglioni, J. M. Duarte, A. Farbin, H. Gupta, L. Hendriks, L. Heinrich, J. Howarth, P. Jawahar, A. Jueid, J. Lastow, A. Leinweber, J. Mamuzic, E. Merényi, A. Morandini, P. Moskvitina, C. Nellist, J.Ngadiuba, B. Ostdiek, M. Pierini, B. Ravina, R. Ruiz de Austri, S. Sekmen, M. Touranakou, M. Vaškevičiūte, R. Vilalta, J. R. Vlimant, R. Verheyen, M. White, E. Wulff, E. Wallin, K. A. Wozniak, Z. Zhang, \\
% \href{https://scipost.org/10.21468/SciPostPhys.12.1.043}{\underline{SciPost Physics 2022}} [\href{https://arxiv.org/pdf/2105.14027.pdf}{arXiv}] [\href{https://github.com/bostdiek/DarkMachines-UnsupervisedChallenge}{code}]
%----------------------------------------------------------------------------------------
%	TALKS
%----------------------------------------------------------------------------------------
\section{Selected talks}
More complete list \href{https://rbvh.github.io/}{here}.

\begin{tabularx}{\linewidth}{ @{}l r@{} }
\underline{HP2 conference, Newcastle, United Kingdom} Invited talk & \hfil  Sep 2022 \\[1.75pt]
\textbf{An overview of the PanScales parton showers} & \\[3.5pt]

\underline{Heidelberg University, Germany} Invited seminar & \hfill Jun 2022 \\[1.75pt]
\textbf{Event Generation and Density Estimation with Surjective Normalizing Flows} & \\[3.5pt]

\underline{Pittsburgh Loopfest conference, United States} Invited talk & \hfill May 2022 \\[1.75pt]
\textbf{The PanScales parton showers for hadron collisions} & \\[3.5pt]

\underline{University of Jyväskylä, Finland} PYTHIA public seminar & \hfill May 2020 \\[1.75pt]
\textbf{Electroweak Corrections in the Vincia Parton Shower} & \\[3.5pt]

\underline{Monash University, Australia} Invited talk & \hfill Mar 2019 \\[1.75pt]
\textbf{Event Generation and Statistical Sampling with Deep Generative Models} &
\end{tabularx}





%----------------------------------------------------------------------------------------
%	SKILLS
%----------------------------------------------------------------------------------------
\section{Skills}
\begin{tabularx}{\linewidth}{@{}l X@{}}
ML frameworks & \normalsize{PyTorch, TensorFlow, dgl} \\
Programming languages & \normalsize{C++, Python (NumPy, Pandas, Matplotlib), Mathematica} \\
Software & \normalsize{LaTeX, git, svn}  \\
\end{tabularx}

\vfill
\center{\footnotesize Last updated: \today}

\end{document}
